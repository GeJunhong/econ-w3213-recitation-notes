\documentclass[11pt]{scrartcl}
\usepackage{dominatrix}
\usepackage{colortbl}
\usepackage{pgfplots}
\newcommand{\jon}{J\'{o}n }
\newcommand{\oneth}{\ensuremath{\frac{1}{3}}}
\newcommand{\twoth}{\ensuremath{\frac{2}{3}}}
\newcommand{\ve}{\varepsilon}
\pgfplotsset{compat=1.9}
\definecolor{light-gray}{gray}{0.75}
\title{Midterm Review Session}
\subject{ECON W3213 Spring 2014 \jon Steinsson}
\author{Linan Qiu, lq2137}
\begin{document}

\maketitle

\begin{abstract}
This is a set of questions to get you \textbf{started} on revision. They are very very basic. Don't expect the midterm to be this easy (just check out the sample finals!) Don't be too happy if you can answer these. After all, these are the \textbf{bare basics}. If you can't answer them, however, you're in deep shit. In that case, as Gandalf says, you shall not pass.

You will also notice that I skipped giving a lot of the formulas. They are a pain in the ass to type, and they're given in your formula sheet. However, I do ask you do derive most of them. Great exercise for revision!
\end{abstract}

\section{Measuring Inflation}

Inflation $\pi_t$ is given by

\[ \pi_t + 1 = \frac{P_{t,a} Q_a + P_{t,c}Q_c}{P_{t-1, a}Q_a + P_{t-1,c}Q_c} \] 

\begin{enumerate}
\item What is a basket of goods?
\item Using the equation above and the little table that we drew for inflation (just read my recitation notes), explain the differences between
\begin{enumerate}
\item Laspeyres Index
\item Paasche Index
\item Fisher Index
\end{enumerate}
\end{enumerate}

\section{Medieval Monetary Economics}

\begin{enumerate}
\item In the money market,
\begin{enumerate}
\item What is money demand? How does that relate to the value of transactions?
\item What is money supply?
\item How do we combine the two above to form the money velocity equation? 
\end{enumerate}
\item What production function do we use? Why only $L_t$?
\item What is the price adjustment equation? What is the intuition behind this equation? 
\item What are the long run dynamics of this model?
\item What are the short run dynamics of this model?
\item What lags what and what adjusts immediately and what eventually settles to what? (If your answer to this is whaaaaaat, goooooood luck :P )
\end{enumerate}

\section{Gold Standard}

\begin{enumerate}
\item Revisiting the definition of money supply, define these and show how they relate to one another
\begin{enumerate}
\item Amount of gold $M_g$
\item Amount of deposits in banks $M_b$
\item Total money supply $M$
\item Show how we get $M=M_g \frac{M_b}{M_g} \frac{M}{M_b}$
\item What is the Gold Coverage Ratio? The Bank Money Multiplier? (Careful, the GCR is kind of an inverse)
\end{enumerate}
\item If A imports from B, which direction is gold flowing?
\item What's the impact on prices of such gold flows? 
\begin{enumerate}
\item What can the government do to the GCR?
\item Should the government do this? Will it be violating the "Rules of the Game" of countries in the gold standard?
\end{enumerate}
\item On hyperinflations
\begin{enumerate}
\item Can governments print money to fund deficit spending?
\item Can it eliminate such increases in money supply using borrowing?
\item For additional practice on this, just refer to your problem set. That's one hardcore pain in the a**.
\end{enumerate}
\end{enumerate}

\section{Great Depression}

\begin{enumerate}
\item What are the 4 causes of the Great Depression?
\item With regard to the Sterling Crisis (which is the 2nd cause. Ha! Freebie!)
\begin{enumerate}
\item How do you solve dynamic games?
\item Who chooses first in dynamic games?
\item What does that little dotted bubble / rectangle around Investor B of the Bank Run game mean?
\item What can governments do to its cost of defending its currency to ensure that investors not run all the time?
\end{enumerate}
\item With regard to the Bank Bailouts
\begin{enumerate}
\item This is moral hazard. What do the banks expect the government to do?
\item Is the promise of "no-bail-out" credible?
\end{enumerate}
\end{enumerate}

\section{Business Cycles}

\begin{enumerate}
\item How do we define output gao $\tilde{Y}_t$
\item For the medieval AD and SRAS
\begin{enumerate}
\item How do we derive AD?
\item How do we derive SRAS?
\item What happens when AD increases? Does the SRAS increase as well? 
\item Trace out what happens for 3 periods for a temporary increase in money supply (hence positive $\Delta \log{M_t}$ for one period
\item Trace out what happens during a permanent increase in money supply
\end{enumerate}
\item What happens during positive trend growth?
\item What is the function of Okun's Law
\item What is the Non Accelerating Inflationary Rate of Unepmloyment
\item How do we derive the Phillip's Curve? With regard to the Phillip's Curve
\begin{enumerate}
\item Does the relationship hold in the long run? If not, what happens to the Phillip's Curve if output gap is not at 0 or if unemployment is not at NAIRU?
\item How do we add adaptive expectation into the model to formalize this intuition?
\item Trace out what happens if $\tilde{Y}_t$ is made to be permanently positive by having a positive $\Delta \log{M_t}$. Then use Okun's Law to translate the effects on output gap to unemployment.
\end{enumerate}
\end{enumerate}

\section{Monetary Policy}

\begin{enumerate}
\item For the IS Curve,
\begin{enumerate}
\item Derive the investment curve
\item Derive the savings curve
\item Combine them to form the IS curve
\end{enumerate}
\item For money demand and supply
\begin{enumerate}
\item Derive money demand, now adding the assumption that money velocity is not constant and is instead related to nominal interest rate
\item What is the money supply?
\item What conclusion do we get from this? Can the central bank now control nominal interest rates?
\end{enumerate}
\item How do we use the Fisher Equation to link nominal interest rates and real interest rates? So can the central bank control real interest rates now?
\item If so, what is the MP curve?
\item Shocks to the IS-MP
\begin{enumerate}
\item What happens if there is a temporary boom?
\item What if the central bank wants to remove any inflationary effects from such a boom?
\item Do the other examples in the recitation notes
\item Derive the AD-AS model.
\item Compare the AD-AS model to the IS-MP / Phillips Curve model and show that they both describe the same thing.
\end{enumerate}
\end{enumerate}

\section{Fiscal Policy}

\begin{enumerate}
\item What is the fiscal multiplier? Define it.
\item What is the fiscal multiplier for each of these scenarios? Trace the entire dynamics between IS-MP and the Phillips Curve.
\begin{enumerate}
\item Base case without monetary response and without liquidity trap
\item Base case with monetary response
\item Hand to mouth consumer
\item Zero lower bound or Liquidity Trap
\item Useful Spending
\item Wasteful Spending
\end{enumerate}
\item Provide empirical evidence for the last two
\end{enumerate}

\section{Bonus Question}

What is \jon's favorite drink?









\end{document}