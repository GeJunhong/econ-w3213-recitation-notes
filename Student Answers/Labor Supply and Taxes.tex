\documentclass[11pt]{scrartcl}
\usepackage{dominatrix}
\usepackage{colortbl}
\usepackage{pgfplots}
\newcommand{\jon}{Jón }
\pgfplotsset{compat=1.9}
\renewcommand\thesubsection{\alph{subsection}}
\definecolor{light-gray}{gray}{0.75}
\title{Answer to Question on Labor Supply and Taxes}
\subject{ECON W3213 Spring 2014 \jon Steinsson}
\author{Linan Qiu, lq2137}
\begin{document}
\maketitle

\section{Question}

Derivation of:

\[\frac{V'(H)}{U'(C)} = w \frac{1- \tau_l}{1+\tau_c}\] 

From \textbf{Lecture Notes 4}

\section{Answer}

We have a utility function

\[U(C) - V(H)\]

The budget constraint (without taxes) used to be this

\[C = wH\]

However, with taxes, the amount of money we spend on consumption is however much we buy $C$ plus the consumption tax on the amount that we bought $\tau_c C$. That makes the new LHS $(1+\tau_c)C$

The still get $wH$ as income, but we get $\tau_l wH$ taxed away. Our disposable income then becomes $(1-\tau_l)wH$ 

Then our new budget constraint (with taxes) is

\[(1+\tau_c)C = (1-\tau_l)wH\]

Rearranging,

\[ C = wH \frac{1+\tau_c}{1-\tau_l} \]

Substituting back into the original utility function (since that's how we solve constrained optimization without using Lagrangians), the utility function becomes

\[ U \left( wH \frac{1+\tau_c}{1-\tau_l} \right) - V(H) \]

Differentiating with respect to $H$ and setting to $0$, we get

\begin{align*}
\left(w\frac{1+\tau_c}{1-\tau_l}\right) U' \left( wH \frac{1+\tau_c}{1-\tau_l} \right) - V'(H) &= 0 \\
\left(w\frac{1+\tau_c}{1-\tau_l} \right) U' (C) - V'(H) &= 0 \\
\frac{V'(H)}{U'(C)} &= w\frac{1-\tau_l}{1+\tau_c}
\end{align*}

We have established a relationship, just like we did with the non-tax case, between $H$ and $w$. This shows how much number of hours worked $H$ will change if we vary $w$. This is then our \textbf{labor supply post taxes}

\jon then proceeded to use a specific utility function

\[\log{C} - \psi \frac{H^{1+\eta}}{1+\eta} \]









\end{document}